\documentclass[12pt]{article}
\usepackage[english]{babel}

\usepackage[top=2.5cm, left=2.5cm, bottom=2.5cm, right=3cm]{geometry}
%\date{\ }

\usepackage{adjustbox}

% \usepackage{caption">caption}
\usepackage[labelfont=bf]{caption}

\usepackage[T1]{fontenc}
\usepackage[utf8]{inputenc}
\usepackage{lmodern}
%\usepackage[a4paper,margin=1cm]{geometry}
\usepackage{babel}

\usepackage{sectsty}


\sectionfont{\centering}

%\usepackage{setspace}
%\doublespacing

% \usepackage{caption">caption}

\usepackage{enumerate}

\usepackage{float}

\usepackage{epsfig,graphicx,graphics,latexsym,amssymb,amsfonts,amsmath,amscd, euscript}
\usepackage{changebar}

\usepackage{graphicx}
\usepackage{tikz}
\usetikzlibrary{positioning}

\usepackage[utf8]{inputenc}

\usepackage{pgfplots}

\usepackage{tikz,tkz-tab}
\usetikzlibrary{arrows}
\pagestyle{myheadings}



\usepackage[final]{pdfpages} 

\newtheorem{definition}{{\bf Definition}}[section]
\newtheorem{theorem}[definition]{{\bf Theorem}}
\newtheorem{maintheorem}[definition]{{\bf Main Theorem}}
\newtheorem{corollary}[definition]{{\bf Corollary}}
\newtheorem{proposition}[definition]{\noindent {\bf Proposition}}
\newtheorem{lemma}[definition]{\noindent {\bf Lemma}}
\newtheorem{fact}[definition]{\noindent {\bf Fact}}
\newtheorem{sublemma}[definition]{\noindent {\bf Sublemma}}
\newtheorem{observation}[definition]{\noindent {\bf Observation}}
\newtheorem{claim}[definition]{\noindent {\bf Claim}}
\newtheorem{question}[definition]{\noindent {\bf Question}}
\newtheorem{questions}[definition]{\noindent {\bf Questions}}
\newtheorem{example}[definition]{\noindent {\bf Example}}
\newtheorem{comment}[definition]{\noindent{\bf Comment}}
\newtheorem{comments}[definition]{\noindent {\bf Comments}}
\newtheorem{problem}[definition]{\noindent {\bf Problem}}
\newtheorem{problems}[definition]{\noindent {\bf Problems}}
\newtheorem{conjecture}[definition]{{\bf Conjecture}}
\newtheorem{remark}[definition]{\noindent {\bf Remark}}
\newtheorem{remarks}[definition]{\noindent {\bf Remarks}}
\newtheorem{notation}[definition]{\noindent {\bf Notation}}


\def\Proof{{\parindent0pt {\bf Proof.\ }}}
\def\Prf{{\parindent0pt {\bf Proof.\ }}}


\def\endproof{\hfill {\kern 6pt\penalty 500
		\raise -0pt\hbox{\vrule \vbox to5pt {\hrule width 5pt
				\vfill\hrule}\vrule}}}
\bibliographystyle{plain}

\usepackage{enumerate}
\newcommand{\R}{\mathbb{R}}
\newcommand{\K}{\mathbb{K}}
\newcommand{\F}{\mathbb{F}}
\newcommand{\N}{\mathbb{N}}
\newcommand{\C}{\mathbb{C}}
\newcommand{\Z}{\mathbb{Z}}
\newcommand{\Q}{\mathbb{Q}}
\newcommand{\J}{\mathbb{J}}

\usepackage{xcolor}


\newcommand{\ds}{\displaystyle}
\newcommand{\s}{\scriptstyle}

\renewcommand{\thefootnote}{}

\def\refname{{\normalsize References}}

\begin{document}
	
	
	
	
	
	
	
	
	\title{\bf{Deux conjectures sur $Ord(v)$
	%, le nombre de posets d'un ensemble \`a $v$ \'el\'ements
}}
	\author{\small Didier GARCIA \\
		\centerline{  {\small Villeurbanne, France}}\\
		%\centerline{$^{b}$  {\small  Department of Mathematics, College of Sciences, King Saud University.}}\\
	}
	\maketitle
	\footnote{
	%\noindent $^{\ast}$ Corresponding author.\\
		{\it E-mail address:}
		 digama@free.fr (D. Garcia).
	}
	
	\vspace{-1.4cm}
	
%
%\begin{abstract}
%J'am\'eliore.
%
%  % keywords are optional
%\bigskip\noindent \textbf{Keywords:} number; order; Euler map.
%\end{abstract}	

 La fonction $Ord$ est  la fonction qui \`a un entier $v$ associe $Ord(v)$, le nombre
d'ordres partiels d'un ensemble à $v$ \'el\'ements.
%notons qu'un ordre partiel est aussi appel\'e {\it poset}. 
Dans A001035-OEIS,  $Ord(v)$ est not\'e $a(v)$. 
La notation $x \ (mod \ n)$ d\'esigne le reste de la division euclidienne de $x$ par $n$, appel\'e aussi le résidu de $x$ modulo $n$. La notation $\varphi$ d\'esigne  l'indicatrice d'Euler.
%La notation $a * b$ d\'esigne le produit $ab$.\\
%Je donne une am\'elioration de la formule internet de A001035-OEIS :
%    $$\mbox{Il existe un entier}\ n \ \mbox{tel que} \ Ord(19)  =9699690*n - 1615151$$    
%  \noindent   en \'etablissant la formule suivante :  
%$$\mbox{Il existe un entier}\ n \ \mbox{tel que}\ Ord(19) = 232792560*n + 163279579.\ \ \  (\spadesuit )$$

\noindent Pour l'instant nous avons les valeurs  de $Ord(v)$ jusqu'\`a  $v = 18$ comme indiqué dans \cite{sloane}.\\
% ainsi que l'\'ecrit Maurice.

Je propose les deux conjectures suivantes.

\begin{conjecture}
Pour tout entier naturel non nul $n$ et tout entier naturel $k$, $Ord(n)$ divise $
Ord(n+k\varphi(Ord(n))$.
\end{conjecture}

\begin{conjecture}
Soit $(u_n)_{n\geq 1}$ une suite d'entiers naturels.  
Les deux conditions suivantes sont \'equivalentes :
\begin{enumerate}[1)]
\item $(u_n)_{n\geq 1}$  v\'erifie les conditions :
\begin{enumerate}[i)]
\item $u_1=1$
\item Pour tout nombre premier $p$, la suite $(u_n \ (mod \ p))_{n\geq 1}$ est 
p\'eriodique de p\'eriode $p-1$.
\end{enumerate} 
\item $u_n = Ord(n)$ pour tout  $n\geq 1$.  
\end{enumerate} 
\end{conjecture}


 



 %Voir le livre de Maurice Pouzet (page 30 environ).

%Calculer $Ord(19)$ peut \^etre un Challenge, et par l\`a m\^eme l'approximer peut \^etre profitable.
%
% 
%\noindent Pour \'etablir $(\spadesuit)$, je montre d'abord que $Ord(19)$ est congru \`a $4$ modulo ($3^2 = 9$)    et  \`a $11$ modulo ($2^4 = 16$) gr\^ace au tableau "Table 1" (page 4); 
% %"congruence.pdf". 
%dans ce tableau :
%
% 
%
%%Ceci comment ?
%%
%%
%%
%%Par prolongement ou extrapolation,  comment dire ?
%
%
%%Je vous propose d'observer le tableau congruence.pdf.
%
%
% - Il y a des lignes de valeurs de $v$ allant de $1$ \`a $18$ et des colonnes de
% valeurs de $j$ allant de $1$ \`a $36$.
%
% 
%
%- \`A l'intersection de la ligne $v$ et de la colonne $j$, il y a la valeur de
%$Ord(v) \ (mod \ j)$ que  j'ai calcul\'e avec Sagemath, en r\'ecup\'erant les valeurs de
%$Ord(v)$ publi\'ees dans \cite{sloane}.
%
% 
%
%- La derni\`ere ligne est $\varphi(j)$ o\`u $\varphi$ d\'esigne  l'indicatrice d'Euler.\\
% %appliqu\'ee \`a $j$.\\
%   
%%( qui est une p\'eriode des r\'esidus modulo $j$ au sens large ainsi que je l'ai d\'ecouvert... et dont j'ai envie d'en mettre une d\'emonstration
%%par \'ecrit pour la validation.
%
%Soit $E$ un ensemble \`a $n$ \'el\'ements. 
%Un espace topologique $E$ est dit de Kolmogorov ou $T_0$ si pour tout couple d'\'eléments distincts $x$ et $y$ de $E$, il existe un ouvert qui contient l'un des deux points mais pas l'autre. 
% Rappelons le r\'esultat suivant :
% \begin{theorem} \label{meme nombre} (\cite{Alexandrov, Birkhoff, Sharp})
% Soit $E$ un ensemble fini non vide \`a $n$ \'el\'ements, alors l'ensemble des $T_0$-topologies sur $E$ et l'ensemble des ordres sur $E$ sont \'equipotents, et par suite ils ont le m\^eme nombre d'\'el\'ements not\'e $T_0(n)$. 
%\end{theorem}
%
%
%- Dans l'article de Borevitch \cite{Bor80}, 
%%en pi\`ece jointe, 
%j'extrais son th\'eor\`eme de p\'eriodicit\'e pour $p$ nombre premier et sa remark 2. 
%% qui stipule que pour $p^s$ la p\'eriode des restes est de $\varphi(p^s) =
%%p^s - p^{s-1}$              et d\'ebute \`a $v = p^{s-1}$.
%
%\begin{theorem} (Borevitch \cite{Bor80})
%Soit $p$ un nombre premier. Soit $k$ et $\ell$ deux entiers naturels v\'erifiant $k\equiv \ell \ (mod \ p-1)$. Alors $T_0(k)\equiv T_0(\ell) \ (mod \ p)$, et alors la suite 
%$(T_0(n)\ (mod \ p))_{n\geq 1}$ est p\'eriodique.
%\end{theorem}
%
%\begin{remark} \label{periode} (Remark 2 de Borevitch \cite{Bor80})
%Soit $p$ un nombre premier et soit $a\geq 1$ un entier. Alors la suite $(T_0(n)\ (mod \ p^a))_{n\geq 1}$ est p\'eriodique seulement \`a partir de $n\geq p^{a-1}$ et la p\'eriode est \'egale \`a $\varphi (p^a) = p^a - p^{a-1}$.
%\end{remark} 
%%
%% Dans les deux cas, la
%%p\'eriode est donn\'ee par l'application de  la fonction $\varphi$.\\
%
%
%Dans ce qui suit, le motif p\'eriodique (\`a visualiser
%verticalement dans les colonnes de $j$ de la Table 1) c'est par
% exemple :
% 
% - La suite finie de $2$ termes ($1$, $0$) des restes de $Ord(v)$ modulo ($j = 3$).
% o\`u l'on remarque au passage que $Ord(2 * n)$ est congru \`a $0$ modulo $3$, 
% i.e. $Ord(2 * n)$ multiple  de $3$.\\
% 
%
%- La suite finie  des $6$ termes ($1$, $3$, $5$, $2$, $3$, $5$) des restes des congruences
%modulo $j = p = 7$ que l'on trouve  verticalement dans  la Table 1.\\
%
%
%- Pour $j = 16 = 2^4$ la p\'eriode est de $\varphi(16) = 8$ et elle commence \`a $v = 2^3 = 8$.\\
%Comme $8 + 8 = 16$ est inf\'erieur  \`a  $19$, nous observons le motif
%p\'eriodique de huit termes  (\`a  lire verticalement sur notre tableau) : ($3$,
% $15$, $15$, $11$, $11$, $7$, $7$, $3$).
%
%Pour $Ord(v = 19)$  modulo  ($j = 16$) nous trouvons le r\'esidu $11$ comme
%prolongement du tableau vers le bas.
%
% - Pour $j = 9 = 3^2$ la p\'eriode est $9 - 3 = 6$ et d\'ebute \`a $v = 3$, c'est encore
% bon car $6 + 3 = 9$ est inf\'erieur  \`a $19$. Donc on observe le motif p\'eriodique ($1$, $3$, $1$, $0$, $4$, $3$) verticalement sur notre tableau.\\
%
%Pour $Ord(19)$ modulo ($j = 9$), nous trouvons le r\'esidu $4$ comme prolongement.
%Par contre, si on prend $p = 5$ et $s = 2$   alors
%on a  $25 = 5^2 = p^s$. La période est de $\varphi(25) = 5^2 - 5 = 20$ et d\'ebute \`a partir de $5$. Or $20 + 5 = 25$  d\'epasse $v = 19$, le motif est trop long (longueur de $20$) pour que l'on puisse en voir la r\'ep\'etition. C'est pour cela que je n'ai pas devin\'e la valeur de $Ord(19)$ modulo $25$. Par contre, dans la Table 1, nous pouvons deviner par prolongation \`a la ligne $v = 19$ des colonnes des valeurs de $j$.
%
% Pour cela on se rappelle qu'en choisissant $p$ premier il y a une p\'eriodicit\'e de $\varphi(p) 
%= p - 1$ de l'apparition des r\'esidus d'apr\`es Remark \ref{periode}.\\
%%Comme vous pouvez le v\'erifier :
%
% \noindent Il s'agit finalement de r\'esoudre le  {syst\`eme  d'\'equations} suivant :\\
% 
%$ \begin{cases}
%Ord(19) &\equiv  11 \ (mod \ 16) \ \mbox{au lieu de prendre} \ Ord(19) \equiv  1 \ (mod \ 2)\\
%Ord(19) &\equiv  4 \ (mod \ 9) \ \mbox{au lieu de prendre} \ Ord(19) \equiv  1 \ (mod \ 3)\\
%Ord(19) &\equiv  4\ (mod \ 5) \\
%
%Ord(19) &\equiv  1\ (mod \ 7) \\
%Ord(19) &\equiv  1\ (mod \ 11) \\
%Ord(19) &\equiv  8\ (mod \ 13) \\
%Ord(19) &\equiv  2\ (mod \ 17) \\
%Ord(19) &\equiv  1\ (mod \ 19) \\
%\end{cases}$
%
%Notons que les diff\'erents $j$ sont deux \`a deux premiers entre eux. Nous pouvons  
%r\'esoudre ce syst\`eme avec Sagemath en utilisant le th\'eor\`eme des restes chinois, avec la fonction {\tt crt}~: 
%\begin{displaymath}
%\tt crt([11,4,4,1,1,8,2,1],[16,9,5,7,11,13,17,19])
%\end{displaymath}
%
%$c$ pour chinois
%
%$r$ pour reste
%
%$t$ pour th\'eor\`eme.\\
%%Cela plaisait \`a Maurice que je fasse le lien entre la Th\'eorie de l'Ordre et
%%
%%le Th\'eor\`eme des restes chinois.
% %AVIS aux Amateurs et aux Professionnels pour la courroie de Transmission !
%\\
%qui donne 163279579. 
%\\
%
%\noindent Nous trouvons que :
%$$Ord(19)  \equiv 163279579 \ (mod \ 232792560 = 16*9*5*7*11*13*17*19).$$ 
%Donc $Ord(19)$ est de la forme annoncée :
%$$Ord(19)=232792560*n + 163279579  \ \ \  (\spadesuit )$$
%
%
%
%
%Nous obtenons ainsi pour $Ord(19)$ une formule plus pr\'ecise que celle de A001035-OEIS d'internet. On peut retrouver cette dernière à partir de $({\spadesuit)}$ en posant $n' = 24\, n + 17$, ce qui conduit à : 
%$Ord(19) = 9699690*n' - 1615151$ (formule de A001035-OEIS).\\
%
% 
%
%% P.S : Il existe une th\'eorie des Motifs en Math\'ematiques.
%%
%% Nous pourrions peut-\^etre faire un lien entre la Th\'eorie de l'Ordre et la
%%Th\'eorie des Motifs ?
%
%
%
%%
%\bigskip\bigskip\bigskip
%
%
%%\noindent Je propose ces deux conjectures.
%%
%%\begin{conjecture}
%%Pour tout entier naturel non nul $n$ et tout entier naturel $k$, $Ord(n)$ divise $
%%Ord(n+k\varphi(Ord(n))$.
%%\end{conjecture}
%%
%%\begin{conjecture}
%%Soit $(u_n)_{n\geq 1}$ une suite d'entiers naturels. \\
%%Alors les deux conditions suivantes sont \'equivalentes :
%%\begin{enumerate}[1)]
%%\item $(u_n)_{n\geq 1}$  v\'erifie les conditions :
%%\begin{enumerate}[i)]
%%\item $u_1=1$
%%\item Pour tout nombre premier $p$, la suite $(u_n \ mod \ p)_{n\geq 1}$ est p\'eriodique de p\'eriode $p-1$.
%%\end{enumerate} 
%%\item $u_n = Ord(n)$ pour tout  $n\geq 1$.  
%%\end{enumerate} 
%%\end{conjecture}
%%
%
%
%\newpage
%\vspace{0.4cm}
%
%	\centering
%	\begin{adjustbox}{width=1\textwidth}
%		\begin{tabular}{c | r | r r r r r r r r r r r r r r r r r r r r r r r r r r r r r r r r r r r r}
%	& $v$\\
%	\hline
%	$j$ & 0 & 0 & 2 & 3 & 4 & 5 & 6 & 7 & 8 & 9 & 10 & 11 & 12 & 13 & 14 & 15 & 16 & 17 & 18 & 19 & 20 & 21 & 22 & 23 & 24 & 25 & 26 & 27 & 28 & 29 & 30 & 31 & 32 & 33 & 34 & 35 & 36\\
%	\hline
%	& 1 & 0 & 1 & 1 & 1 & 1 & 1 & 1 & 1 & 1 & 1 & 1 & 1 & 1 & 1 & 1 & 1 & 1 & 1 & 1 & 1 & 1 & 1 & 1 & 1 & 1 & 1 & 1 & 1 & 1 & 1 & 1 & 1 & 1 & 1 & 1 & 1\\
%	& 2 & 0 & 1 & 0 & 3 & 3 & 3 & 3 & 3 & 3 & 3 & 3 & 3 & 3 & 3 & 3 & 3 & 3 & 3 & 3 & 3 & 3 & 3 & 3 & 3 & 3 & 3 & 3 & 3 & 3 & 3 & 3 & 3 & 3 & 3 & 3 & 3\\
%	& 3 & 0 & 1 & 1 & 3 & 4 & 1 & 5 & 3 & 1 & 9 & 8 & 7 & 6 & 5 & 4 & 3 & 2 & 1 & 0 & 19 & 19 & 19 & 19 & 19 & 19 & 19 & 19 & 19 & 19 & 19 & 19 & 19 & 19 & 19 & 19 & 19\\
%	& 4 & 0 & 1 & 0 & 3 & 4 & 3 & 2 & 3 & 3 & 9 & 10 & 3 & 11 & 9 & 9 & 11 & 15 & 3 & 10 & 19 & 9 & 21 & 12 & 3 & 19 & 11 & 3 & 23 & 16 & 9 & 2 & 27 & 21 & 15 & 9 & 3\\
%	& 5 & 0 & 1 & 1 & 3 & 1 & 1 & 3 & 7 & 1 & 1 & 7 & 7 & 6 & 3 & 1 & 7 & 15 & 1 & 13 & 11 & 10 & 7 & 22 & 7 & 6 & 19 & 19 & 3 & 26 & 1 & 15 & 7 & 7 & 15 & 31 & 19\\
%	& 6 & 0 & 1 & 0 & 3 & 3 & 3 & 5 & 7 & 0 & 3 & 3 & 3 & 10 & 5 & 3 & 7 & 7 & 9 & 6 & 3 & 12 & 3 & 4 & 15 & 23 & 23 & 18 & 19 & 16 & 3 & 9 & 7 & 3 & 7 & 33 & 27\\
%	& 7 & 0 & 1 & 1 & 3 & 4 & 1 & 1 & 3 & 4 & 9 & 10 & 7 & 8 & 1 & 4 & 3 & 16 & 13 & 3 & 19 & 1 & 21 & 14 & 19 & 9 & 21 & 22 & 15 & 13 & 19 & 12 & 3 & 10 & 33 & 29 & 31\\
%	& 8 & 0 & 1 & 0 & 3 & 4 & 3 & 3 & 3 & 3 & 9 & 10 & 3 & 9 & 3 & 9 & 3 & 15 & 3 & 2 & 19 & 3 & 21 & 16 & 3 & 4 & 9 & 21 & 3 & 2 & 9 & 19 & 19 & 21 & 15 & 24 & 3\\
%	& 9 & 0 & 1 & 1 & 3 & 1 & 1 & 5 & 7 & 1 & 1 & 1 & 7 & 0 & 5 & 1 & 15 & 5 & 1 & 18 & 11 & 19 & 1 & 20 & 7 & 11 & 13 & 1 & 19 & 25 & 1 & 0 & 15 & 1 & 5 & 26 & 19\\
%	& 10 & 0 & 1 & 0 & 3 & 3 & 3 & 2 & 7 & 3 & 3 & 6 & 3 & 8 & 9 & 3 & 15 & 15 & 3 & 10 & 3 & 9 & 17 & 7 & 15 & 8 & 21 & 12 & 23 & 12 & 3 & 18 & 15 & 6 & 15 & 23 & 3\\
%	& 11 & 0 & 1 & 1 & 3 & 4 & 1 & 3 & 3 & 1 & 9 & 1 & 7 & 12 & 3 & 4 & 11 & 0 & 1 & 2 & 19 & 10 & 1 & 7 & 19 & 24 & 25 & 10 & 3 & 13 & 19 & 29 & 11 & 1 & 17 & 24 & 19\\
%	& 12 & 0 & 1 & 0 & 3 & 4 & 3 & 5 & 3 & 0 & 9 & 3 & 3 & 6 & 5 & 9 & 11 & 8 & 9 & 8 & 19 & 12 & 3 & 7 & 3 & 24 & 19 & 18 & 19 & 3 & 9 & 14 & 27 & 3 & 25 & 19 & 27\\
%	& 13 & 0 & 1 & 1 & 3 & 1 & 1 & 1 & 7 & 4 & 1 & 8 & 7 & 1 & 1 & 1 & 7 & 8 & 13 & 12 & 11 & 1 & 19 & 2 & 7 & 1 & 1 & 22 & 15 & 24 & 1 & 17 & 23 & 19 & 25 & 1 & 31\\
%	& 14 & 0 & 1 & 0 & 3 & 3 & 3 & 3 & 7 & 3 & 3 & 10 & 3 & 3 & 3 & 3 & 7 & 5 & 3 & 8 & 3 & 3 & 21 & 11 & 15 & 8 & 3 & 12 & 3 & 22 & 3 & 13 & 23 & 21 & 5 & 3 & 3\\
%	& 15 & 0 & 1 & 1 & 3 & 4 & 1 & 5 & 3 & 1 & 9 & 7 & 7 & 6 & 5 & 4 & 3 & 14 & 1 & 14 & 19 & 19 & 7 & 21 & 19 & 14 & 19 & 10 & 19 & 17 & 19 & 21 & 19 & 7 & 31 & 19 & 19\\
%	& 16 & 0 & 1 & 0 & 3 & 4 & 3 & 2 & 3 & 3 & 9 & 3 & 3 & 11 & 9 & 9 & 3 & 3 & 3 & 2 & 19 & 9 & 3 & 16 & 3 & 4 & 11 & 21 & 23 & 8 & 9 & 1 & 3 & 3 & 3 & 9 & 3\\
%	& 17 & 0 & 1 & 1 & 3 & 1 & 1 & 3 & 7 & 1 & 1 & 10 & 7 & 6 & 3 & 1 & 15 & 1 & 1 & 14 & 11 & 10 & 21 & 20 & 7 & 1 & 19 & 1 & 3 & 16 & 1 & 23 & 31 & 10 & 1 & 31 & 19\\
%	& 18 & 0 & 1 & 0 & 3 & 3 & 3 & 5 & 7 & 0 & 3 & 10 & 3 & 10 & 5 & 3 & 15 & 3 & 9 & 0 & 3 & 12 & 21 & 20 & 15 & 23 & 23 & 18 & 19 & 2 & 3 & 6 & 31 & 21 & 3 & 33 & 27\\
%	\hline
%	$\varphi(j)$ & 0 & 0 & 1 & 2 & 2 & 4 & 2 & 6 & 4 & 6 & 4 & 10 & 4 & 12 & 6 & 8 & 8 & 16 & 6 & 18 & 8 & 12 & 10 & 22 & 8 & 20 & 12 & 18 & 12 & 28 & 8 & 30 & 16 & 20 & 16 & 24 & 12
%	\end{tabular}
%	\end{adjustbox}
%	\vspace{0.2cm}
%	\captionof{table}{Valeurs de $Ord(v) \ (mod \ {p-1})$.}
%
%%\includepdf[pages={1-1}]{congruence.pdf}
%
%\begin{thebibliography}{10}\label{bibliography}
%\bibitem{Alexandrov} P.~S.~Aleksandrov, Combinatorial topology, Vol. 1, Graylock, Rochester, N. Y.,
%1956.
%\bibitem{Birkhoff} G. Birkhoff, Lattice Theory. Revised edition. Amer. Math. Soc. Colloq. Publ., Vol. 25
%American Mathematical Society, New York, 1948. xiii+283 pp.
%\bibitem{Bor80} Z.~I.~Borevich,  
%%On the periodicity of residues of the number of finite labeled topologies. 
%%(Russian)
%%Modules and linear groups
%%Zap. Nauchn. Sem. Leningrad. Otdel. Mat. Inst. Steklov. (LOMI)103(1980), 5--12, 155.
%On the periodicity of residues of the number of finite labeled topologies. 
%Zap. Nauchn. Sem. Leningrad. Otdel. Mat. Inst. Steklov. (LOMI) 103 (1980), 5--12, 155.
%\bibitem{Sharp} H. Sharp, Jr., Quasi-orderings and topologies on finite sets.
%Proc. Amer. Math. Soc.17(1966), 1344--1349.
%\bibitem{sloane} N.~Sloane, The On-Line Encyclopedia of Integer Sequences, oeis.org. 
%\end{thebibliography}
%\end{document}
%
%
%
%
%\documentclass[12pt]{article}
%\usepackage[english]{babel}
%
%\usepackage[top=2.5cm, left=2.5cm, bottom=2.5cm, right=3cm]{geometry}
%%\date{\ }
%
%\usepackage{adjustbox}
%
%% \usepackage{caption">caption}
%\usepackage[labelfont=bf]{caption}
%
%\usepackage[T1]{fontenc}
%\usepackage[utf8]{inputenc}
%\usepackage{lmodern}
%%\usepackage[a4paper,margin=1cm]{geometry}
%\usepackage{babel}
%
%\usepackage{sectsty}
%
%
%\sectionfont{\centering}
%
%%\usepackage{setspace}
%%\doublespacing
%
%% \usepackage{caption">caption}
%
%\usepackage{enumerate}
%
%\usepackage{float}
%
%\usepackage{epsfig,graphicx,graphics,latexsym,amssymb,amsfonts,amsmath,amscd, euscript}
%\usepackage{changebar}
%
%\usepackage{graphicx}
%\usepackage{tikz}
%\usetikzlibrary{positioning}
%
%\usepackage[utf8]{inputenc}
%
%\usepackage{pgfplots}
%
%\usepackage{tikz,tkz-tab}
%\usetikzlibrary{arrows}
%\pagestyle{myheadings}
%
%\usepackage[final]{pdfpages} 
%
%\newtheorem{definition}{{\bf Definition}}[section]
%\newtheorem{theorem}[definition]{{\bf Theorem}}
%\newtheorem{maintheorem}[definition]{{\bf Main Theorem}}
%\newtheorem{corollary}[definition]{{\bf Corollary}}
%\newtheorem{proposition}[definition]{\noindent {\bf Proposition}}
%\newtheorem{lemma}[definition]{\noindent {\bf Lemma}}
%\newtheorem{fact}[definition]{\noindent {\bf Fact}}
%\newtheorem{sublemma}[definition]{\noindent {\bf Sublemma}}
%\newtheorem{observation}[definition]{\noindent {\bf Observation}}
%\newtheorem{claim}[definition]{\noindent {\bf Claim}}
%\newtheorem{question}[definition]{\noindent {\bf Question}}
%\newtheorem{questions}[definition]{\noindent {\bf Questions}}
%\newtheorem{example}[definition]{\noindent {\bf Example}}
%\newtheorem{comment}[definition]{\noindent{\bf Comment}}
%\newtheorem{comments}[definition]{\noindent {\bf Comments}}
%\newtheorem{problem}[definition]{\noindent {\bf Problem}}
%\newtheorem{problems}[definition]{\noindent {\bf Problems}}
%\newtheorem{conjecture}[definition]{{\bf Conjecture}}
%\newtheorem{remark}[definition]{\noindent {\bf Remark}}
%\newtheorem{remarks}[definition]{\noindent {\bf Remarks}}
%\newtheorem{notation}[definition]{\noindent {\bf Notation}}
%
%
%\def\Proof{{\parindent0pt {\bf Proof.\ }}}
%\def\Prf{{\parindent0pt {\bf Proof.\ }}}
%
%
%\def\endproof{\hfill {\kern 6pt\penalty 500
%		\raise -0pt\hbox{\vrule \vbox to5pt {\hrule width 5pt
%				\vfill\hrule}\vrule}}}
%\bibliographystyle{plain}
%
%\usepackage{enumerate}
%\newcommand{\R}{\mathbb{R}}
%\newcommand{\K}{\mathbb{K}}
%\newcommand{\F}{\mathbb{F}}
%\newcommand{\N}{\mathbb{N}}
%\newcommand{\C}{\mathbb{C}}
%\newcommand{\Z}{\mathbb{Z}}
%\newcommand{\Q}{\mathbb{Q}}
%\newcommand{\J}{\mathbb{J}}
%
%\usepackage{xcolor}
%
%
%\newcommand{\ds}{\displaystyle}
%\newcommand{\s}{\scriptstyle}
%
%\renewcommand{\thefootnote}{}
%
%\def\refname{{\normalsize References}}
%
%\begin{document}
%	
%	
%	
%	
%	
%	
%	
%	
%	\title{\bf{Am\'elioration de la formule A001035-OEIS  donnant le nombre de posets d'un ensemble \`a 19 \'el\'ements
%}}
%	\author{\small Didier GARCIA \\
%		\centerline{  {\small Villeurbanne, France}}\\
%		%\centerline{$^{b}$  {\small  Department of Mathematics, College of Sciences, King Saud University.}}\\
%	}
%	\maketitle
%	\footnote{
%	%\noindent $^{\ast}$ Corresponding author.\\
%		{\it E-mail address:}
%		 digama@free.fr (D. Garcia).
%	}
%	
%	\vspace{-1.4cm}
%	
%%
%%\begin{abstract}
%%J'am\'eliore.
%%
%%  % keywords are optional
%%\bigskip\noindent \textbf{Keywords:} number; order; Euler map.
%%\end{abstract}	
%
% La fonction $Ord$ est  la fonction qui \`a un entier $v$ associe $Ord(v)$, le nombre
%d'ordres partiels d'un ensemble à $v$ \'el\'ements, notons qu'un ordre partiel est aussi appel\'e {\it poset}. Dans A001035-OEIS,  $Ord(v)$ est not\'e $a(v)$. La notation $a \ (mod \ b)$ d\'esigne le reste de la division euclidienne de $a$ par $b$. La notation $a * b$ d\'esigne le produit $ab$.\\
%Je donne une am\'elioration de la formule internet de A001035-OEIS :
%    $$\mbox{Il existe un entier}\ n \ \mbox{tel que} \ Ord(19)  =9699690*n - 1615151$$    
%  \noindent   en \'etablissant la formule suivante :  
%$$\mbox{Il existe un entier}\ n \ \mbox{tel que}\ Ord(19) = 232792560*n + 163279579.\ \ \  (\spadesuit )$$
%
%
%
%\noindent Pour l'instant nous avons les valeurs  de $Ord(v)$ jusqu'\`a  $v = 18$ comme indiqué dans \cite{sloane}.
%% ainsi que l'\'ecrit Maurice.
%
% 
%\noindent Je propose ces deux conjectures.
%
%\begin{conjecture}
%Pour tout entier naturel non nul $n$ et tout entier naturel $k$, $Ord(n)$ divise $
%Ord(n+k\varphi(Ord(n))$.
%\end{conjecture}
%
%\begin{conjecture}
%Soit $(u_n)_{n\geq 1}$ une suite d'entiers naturels. \\
%Alors les deux conditions suivantes sont \'equivalentes :
%\begin{enumerate}[1)]
%\item $(u_n)_{n\geq 1}$  v\'erifie les conditions :
%\begin{enumerate}[i)]
%\item $u_1=1$
%\item Pour tout nombre premier $p$, la suite $(u_n \ mod \ p)_{n\geq 1}$ est p\'eriodique de p\'eriode $p-1$.
%\end{enumerate} 
%\item $u_n = Ord(n)$ pour tout  $n\geq 1$.  
%\end{enumerate} 
%\end{conjecture}
%
%
%
% %Voir le livre de Maurice Pouzet (page 30 environ).
%
%%Calculer $Ord(19)$ peut \^etre un Challenge, et par l\`a m\^eme l'approximer peut \^etre profitable.
%%
%% 
%%\noindent Pour \'etablir $(\spadesuit)$, je montre d'abord que $Ord(19)$ est congru \`a $4$ modulo ($3^2 = 9$)    et  \`a $11$ modulo ($2^4 = 16$) gr\^ace au tableau "Table 1" (page 4). 
%% %"congruence.pdf". 
%%Dans Table 1 :
%%
%% 
%%
%%%Ceci comment ?
%%%
%%%
%%%
%%%Par prolongement ou extrapolation,  comment dire ?
%%
%%
%%%Je vous propose d'observer le tableau congruence.pdf.
%%
%%
%% - Il y a des lignes de valeurs de $v$ allant de $1$ \`a $18$ et des colonnes de
%% valeurs de $j$ allant de $2$ \`a $36$.
%%
%% 
%%
%%- \`A l'intersection de la ligne $v$ et de la colonne $j$, il y a la valeur de
%%$Ord(v) \ (mod \ j)$ que j'ai calcul\'e avec Sagemath, en r\'ecup\'erant les valeurs de
%%$Ord(v)$ publi\'ees dans \cite{sloane}.
%%
%% 
%%
%%- La derni\`ere ligne est $\varphi(j)$ o\`u la fonction $\varphi$ est  l'indicatrice d'Euler. 
%%%appliqu\'ee \`a $j$.\\
%%   
%%%( qui est une p\'eriode des r\'esidus modulo $j$ au sens large ainsi que je l'ai d\'ecouvert... et dont j'ai envie d'en mettre une d\'emonstration
%%%par \'ecrit pour la validation.
%%
%%Soit $E$ un ensemble \`a $n$ \'el\'ements. 
%%Un espace topologique $E$ est dit de Kolmogorov ou $T_0$ si pour tout couple d'\'eléments distincts $x$ et $y$ de $E$, il existe un ouvert qui contient l'un des deux points mais pas l'autre. 
%% Rappelons le r\'esultat suivant :
%% \begin{theorem} \label{meme nombre} (\cite{Alexandrov, Birkhoff, Sharp})
%% Soit $E$ un ensemble fini non vide \`a $n$ \'el\'ements, alors l'ensemble des $T_0$-topologies sur $E$ et l'ensemble des ordres sur $E$ sont \'equipotents, et par suite ils ont le m\^eme nombre d'\'el\'ements not\'e $T_0(n)$. 
%%\end{theorem}
%%
%%
%%- Dans l'article de Borevitch \cite{Bor80}, 
%%%en pi\`ece jointe, 
%%j'extrais son th\'eor\`eme de p\'eriodicit\'e pour $p$ nombre premier et sa remark 2. 
%%% qui stipule que pour $p^s$ la p\'eriode des restes est de $\varphi(p^s) =
%%%p^s - p^{s-1}$              et d\'ebute \`a $v = p^{s-1}$.
%%
%%\begin{theorem} (Borevitch \cite{Bor80})
%%Soit $p$ un nombre premier. Soit $k$ et $\ell$ deux entiers naturels v\'erifiant $k\equiv \ell \ (mod \ p-1)$. Alors $T_0(k)\equiv T_0(\ell) \ (mod \ p)$, et alors la suite 
%%$(T_0(n)\ (mod \ p))_{n\geq 1}$ est p\'eriodique.
%%\end{theorem}
%%
%%\begin{remark} \label{periode} (Remark 2 de Borevitch \cite{Bor80})
%%Soit $p$ un nombre premier et soit $a\geq 1$ un entier. Alors la suite $(T_0(n)\ (mod \ p^a))_{n\geq 1}$ est p\'eriodique seulement \`a partir de $n\geq p^{a-1}$ et la p\'eriode est \'egale \`a $\varphi (p^a) = p^a - p^{a-1}$.
%%\end{remark} 
%%%
%%% Dans les deux cas, la
%%%p\'eriode est donn\'ee par l'application de  la fonction $\varphi$.\\
%%
%%
%%Dans ce qui suit, le motif p\'eriodique (\`a visualiser
%%verticalement dans les colonnes de $j$ de  Table 1) c'est par
%% exemple :
%% 
%% - La suite finie de $2$ termes ($1$, $0$) des restes de $Ord(v)$ modulo ($j = 3$).
%% o\`u l'on remarque au passage que $Ord(2n)$ est congru \`a $0$ modulo $3$, 
%% i.e. $Ord(2n)$ multiple  de $3$.\\
%% 
%%
%%- La suite finie  des $6$ termes ($1$, $3$, $5$, $2$, $3$, $5$) des restes des congruences
%%modulo $j = p = 7$ que l'on trouve  verticalement dans   Table 1.\\
%%
%%
%%- Pour $j = 16 = 2^4$ la p\'eriode est de $\varphi(16) = 8$ et elle commence \`a $v = 2^3 = 8$.\\
%%Comme $8 + 8 = 16$ est inf\'erieur  \`a  $19$, nous observons le motif
%%p\'eriodique de huit termes  (\`a  lire verticalement sur notre tableau) : ($3$,
%% $15$, $15$, $11$, $11$, $7$, $7$, $3$).
%%
%%Pour $Ord(19)$  modulo  ($j = 16$) nous trouvons le reste $11$ comme
%%prolongement du tableau vers le bas.
%%
%% - Pour $j = 9 = 3^2$ la p\'eriode est $9 - 3 = 6$ et d\'ebute \`a $v = 3$, c'est encore
%% bon car $6 + 3 = 9$ est inf\'erieur  \`a $19$. Donc on observe le motif p\'eriodique ($1$ $3$ $1$ $0$ $4$ $3$) verticalement sur notre tableau.\\
%%
%%Pour $Ord(19)$ modulo ($j = 9$), nous trouvons le reste $4$ comme prolongement.
%%Par contre, si on prend $p = 5$ et $s = 2$   alors
%%on a  $25 = 5^2 = p^s$. La période est de $\varphi(25) = 5^2 - 5 = 20$ et d\'ebute \`a partir de $5$. Or $20 + 5 = 25$  d\'epasse $v = 19$, le motif est trop long (longueur de $20$) pour que l'on puisse en voir la r\'ep\'etition. C'est pour cela que je n'ai pas devin\'e la valeur de $Ord(19)$ modulo $25$. Par contre, dans  Table 1, nous pouvons deviner par prolongation \`a la ligne $v = 19$ des colonnes des valeurs de $j$.
%%
%% Pour cela on se rappelle qu'en choisissant $p$ premier il y a une p\'eriodicit\'e de $\varphi(p) 
%%= p - 1$ de l'apparition des restes d'apr\`es Remark \ref{periode}.\\
%%%Comme vous pouvez le v\'erifier :
%%
%% \noindent Il s'agit finalement de r\'esoudre le  {syst\`eme  d'\'equations} suivant :\\
%% 
%%$ \begin{cases}
%%Ord(19) &\equiv  11 \ (mod \ 16) \ \mbox{au lieu de prendre} \ Ord(19) \equiv  1 \ (mod \ 2)\\
%%Ord(19) &\equiv  4 \ (mod \ 9) \ \mbox{au lieu de prendre} \ Ord(19) \equiv  1 \ (mod \ 3)\\
%%Ord(19) &\equiv  4\ (mod \ 5) \\
%%
%%Ord(19) &\equiv  1\ (mod \ 7) \\
%%Ord(19) &\equiv  1\ (mod \ 11) \\
%%Ord(19) &\equiv  8\ (mod \ 13) \\
%%Ord(19) &\equiv  2\ (mod \ 17) \\
%%Ord(19) &\equiv  1\ (mod \ 19) \\
%%\end{cases}$
%%
%%Notons que les diff\'erents $j$ sont deux \`a deux premiers entre eux. Nous pouvons  
%%r\'esoudre ce syst\`eme avec Sagemath en utilisant le th\'eor\`eme des restes chinois.
%%%Cela plaisait \`a Maurice que je fasse le lien entre la Th\'eorie de l'Ordre et
%%%
%%%le Th\'eor\`eme des restes chinois.
%% %AVIS aux Amateurs et aux Professionnels pour la courroie de Transmission !
%%Nous pouvons ainsi avoir une formule plus pr\'ecise de $Ord(19)$ que celle de A001035-OEIS d'internet.
%%
%% 
%%
%%% P.S : Il existe une th\'eorie des Motifs en Math\'ematiques.
%%%
%%% Nous pourrions peut-\^etre faire un lien entre la Th\'eorie de l'Ordre et la
%%%Th\'eorie des Motifs ?
%%
%%
%%
%%\noindent Nous trouvons que :
%%$$Ord(19)  \equiv 163279579 \ (mod \ 232792560 = 16*9*5*7*11*13*17*19).$$ 
%%Donc $Ord(19)$ est de la forme :
%%$$Ord(19)=232792560*n + 163279579$$  qui am\'eliore
%%$Ord(19) = 9699690*n - 1615151$ (de A001035-OEIS).\\
%%
%%
%%\noindent V\'erification de la cohérence des deux formules.\\
%%
%%\noindent Posons $b:=16*9*5*7*11*13*17*19$.\\
%%
%%\noindent On calcule $163279579$ modulo $9699690$, on trouve  $8084539$.
%%
%% $8084539 + 1615151 = 9699690$.
%%
%%C'est coh\'erent, $163279579$ \'etant le r\'esultat de la ligne de commande Sagemath:
%%
%%$a(19) \ (mod \ b) =crt([11,4,4,1,1,8,2,1],[16,9,5,7,11,13,17,19]);a(19)$
%%
%%qui r\'esoud notre syst\`eme, avec la fonction crt de sagemath : 
%%
%%$c$ pour chinois
%%
%%$r$ pour reste
%%
%%$t$ pour th\'eor\`eme.\\
%%
%%
%%\noindent Je propose les deux conjectures suivantes.
%%
%%\begin{conjecture}
%%Pour tout entier naturel non nul $n$ et tout entier naturel $k$, $Ord(n)$ divise $
%%Ord(n+k\varphi(Ord(n))$.
%%\end{conjecture}
%%
%%\begin{conjecture}
%%Soit $(u_n)_{n\geq 1}$ une suite d'entiers naturels.  
%%Les deux conditions suivantes sont \'equivalentes :
%%\begin{enumerate}[1)]
%%\item $(u_n)_{n\geq 1}$  v\'erifie les conditions :
%%\begin{enumerate}[i)]
%%\item $u_1=1$
%%\item Pour tout nombre premier $p$, la suite $(u_n \ mod \ p)_{n\geq 1}$ est p\'eriodique de p\'eriode $p-1$.
%%\end{enumerate} 
%%\item $u_n = Ord(n)$ pour tout  $n\geq 1$.  
%%\end{enumerate} 
%%\end{conjecture}
%%
%%
%%
%%\newpage
%%\vspace{0.4cm}
%%
%%	\centering
%%	\begin{adjustbox}{width=1\textwidth}
%%		\begin{tabular}{c | r | r r r r r r r r r r r r r r r r r r r r r r r r r r r r r r r r r r r r}
%%	& $v$\\
%%	\hline
%%	$j$ & 0 & 0 & 2 & 3 & 4 & 5 & 6 & 7 & 8 & 9 & 10 & 11 & 12 & 13 & 14 & 15 & 16 & 17 & 18 & 19 & 20 & 21 & 22 & 23 & 24 & 25 & 26 & 27 & 28 & 29 & 30 & 31 & 32 & 33 & 34 & 35 & 36\\
%%	\hline
%%	& 1 & 0 & 1 & 1 & 1 & 1 & 1 & 1 & 1 & 1 & 1 & 1 & 1 & 1 & 1 & 1 & 1 & 1 & 1 & 1 & 1 & 1 & 1 & 1 & 1 & 1 & 1 & 1 & 1 & 1 & 1 & 1 & 1 & 1 & 1 & 1 & 1\\
%%	& 2 & 0 & 1 & 0 & 3 & 3 & 3 & 3 & 3 & 3 & 3 & 3 & 3 & 3 & 3 & 3 & 3 & 3 & 3 & 3 & 3 & 3 & 3 & 3 & 3 & 3 & 3 & 3 & 3 & 3 & 3 & 3 & 3 & 3 & 3 & 3 & 3\\
%%	& 3 & 0 & 1 & 1 & 3 & 4 & 1 & 5 & 3 & 1 & 9 & 8 & 7 & 6 & 5 & 4 & 3 & 2 & 1 & 0 & 19 & 19 & 19 & 19 & 19 & 19 & 19 & 19 & 19 & 19 & 19 & 19 & 19 & 19 & 19 & 19 & 19\\
%%	& 4 & 0 & 1 & 0 & 3 & 4 & 3 & 2 & 3 & 3 & 9 & 10 & 3 & 11 & 9 & 9 & 11 & 15 & 3 & 10 & 19 & 9 & 21 & 12 & 3 & 19 & 11 & 3 & 23 & 16 & 9 & 2 & 27 & 21 & 15 & 9 & 3\\
%%	& 5 & 0 & 1 & 1 & 3 & 1 & 1 & 3 & 7 & 1 & 1 & 7 & 7 & 6 & 3 & 1 & 7 & 15 & 1 & 13 & 11 & 10 & 7 & 22 & 7 & 6 & 19 & 19 & 3 & 26 & 1 & 15 & 7 & 7 & 15 & 31 & 19\\
%%	& 6 & 0 & 1 & 0 & 3 & 3 & 3 & 5 & 7 & 0 & 3 & 3 & 3 & 10 & 5 & 3 & 7 & 7 & 9 & 6 & 3 & 12 & 3 & 4 & 15 & 23 & 23 & 18 & 19 & 16 & 3 & 9 & 7 & 3 & 7 & 33 & 27\\
%%	& 7 & 0 & 1 & 1 & 3 & 4 & 1 & 1 & 3 & 4 & 9 & 10 & 7 & 8 & 1 & 4 & 3 & 16 & 13 & 3 & 19 & 1 & 21 & 14 & 19 & 9 & 21 & 22 & 15 & 13 & 19 & 12 & 3 & 10 & 33 & 29 & 31\\
%%	& 8 & 0 & 1 & 0 & 3 & 4 & 3 & 3 & 3 & 3 & 9 & 10 & 3 & 9 & 3 & 9 & 3 & 15 & 3 & 2 & 19 & 3 & 21 & 16 & 3 & 4 & 9 & 21 & 3 & 2 & 9 & 19 & 19 & 21 & 15 & 24 & 3\\
%%	& 9 & 0 & 1 & 1 & 3 & 1 & 1 & 5 & 7 & 1 & 1 & 1 & 7 & 0 & 5 & 1 & 15 & 5 & 1 & 18 & 11 & 19 & 1 & 20 & 7 & 11 & 13 & 1 & 19 & 25 & 1 & 0 & 15 & 1 & 5 & 26 & 19\\
%%	& 10 & 0 & 1 & 0 & 3 & 3 & 3 & 2 & 7 & 3 & 3 & 6 & 3 & 8 & 9 & 3 & 15 & 15 & 3 & 10 & 3 & 9 & 17 & 7 & 15 & 8 & 21 & 12 & 23 & 12 & 3 & 18 & 15 & 6 & 15 & 23 & 3\\
%%	& 11 & 0 & 1 & 1 & 3 & 4 & 1 & 3 & 3 & 1 & 9 & 1 & 7 & 12 & 3 & 4 & 11 & 0 & 1 & 2 & 19 & 10 & 1 & 7 & 19 & 24 & 25 & 10 & 3 & 13 & 19 & 29 & 11 & 1 & 17 & 24 & 19\\
%%	& 12 & 0 & 1 & 0 & 3 & 4 & 3 & 5 & 3 & 0 & 9 & 3 & 3 & 6 & 5 & 9 & 11 & 8 & 9 & 8 & 19 & 12 & 3 & 7 & 3 & 24 & 19 & 18 & 19 & 3 & 9 & 14 & 27 & 3 & 25 & 19 & 27\\
%%	& 13 & 0 & 1 & 1 & 3 & 1 & 1 & 1 & 7 & 4 & 1 & 8 & 7 & 1 & 1 & 1 & 7 & 8 & 13 & 12 & 11 & 1 & 19 & 2 & 7 & 1 & 1 & 22 & 15 & 24 & 1 & 17 & 23 & 19 & 25 & 1 & 31\\
%%	& 14 & 0 & 1 & 0 & 3 & 3 & 3 & 3 & 7 & 3 & 3 & 10 & 3 & 3 & 3 & 3 & 7 & 5 & 3 & 8 & 3 & 3 & 21 & 11 & 15 & 8 & 3 & 12 & 3 & 22 & 3 & 13 & 23 & 21 & 5 & 3 & 3\\
%%	& 15 & 0 & 1 & 1 & 3 & 4 & 1 & 5 & 3 & 1 & 9 & 7 & 7 & 6 & 5 & 4 & 3 & 14 & 1 & 14 & 19 & 19 & 7 & 21 & 19 & 14 & 19 & 10 & 19 & 17 & 19 & 21 & 19 & 7 & 31 & 19 & 19\\
%%	& 16 & 0 & 1 & 0 & 3 & 4 & 3 & 2 & 3 & 3 & 9 & 3 & 3 & 11 & 9 & 9 & 3 & 3 & 3 & 2 & 19 & 9 & 3 & 16 & 3 & 4 & 11 & 21 & 23 & 8 & 9 & 1 & 3 & 3 & 3 & 9 & 3\\
%%	& 17 & 0 & 1 & 1 & 3 & 1 & 1 & 3 & 7 & 1 & 1 & 10 & 7 & 6 & 3 & 1 & 15 & 1 & 1 & 14 & 11 & 10 & 21 & 20 & 7 & 1 & 19 & 1 & 3 & 16 & 1 & 23 & 31 & 10 & 1 & 31 & 19\\
%%	& 18 & 0 & 1 & 0 & 3 & 3 & 3 & 5 & 7 & 0 & 3 & 10 & 3 & 10 & 5 & 3 & 15 & 3 & 9 & 0 & 3 & 12 & 21 & 20 & 15 & 23 & 23 & 18 & 19 & 2 & 3 & 6 & 31 & 21 & 3 & 33 & 27\\
%%	\hline
%%	$\Phi(j)$ & 0 & 0 & 1 & 2 & 2 & 4 & 2 & 6 & 4 & 6 & 4 & 10 & 4 & 12 & 6 & 8 & 8 & 16 & 6 & 18 & 8 & 12 & 10 & 22 & 8 & 20 & 12 & 18 & 12 & 28 & 8 & 30 & 16 & 20 & 16 & 24 & 12
%%	\end{tabular}
%%	\end{adjustbox}
%%	\vspace{0.2cm}
%%	\captionof{table}{Valeurs de $Ord(v) \ (mod \ {p-1})$.}
%
%%\includepdf[pages={1-1}]{congruence.pdf}

\begin{thebibliography}{10}\label{bibliography}
%\bibitem{Alexandrov} P.~S.~Aleksandrov, Combinatorial topology, Vol. 1, Graylock, Rochester, N. Y.,
%1956.
%\bibitem{Birkhoff} G. Birkhoff, Lattice Theory. Revised edition. Amer. Math. Soc. Colloq. Publ., Vol. 25
%American Mathematical Society, New York, 1948. xiii+283 pp.
%\bibitem{Bor80} Z.~I.~Borevich,  
%On the periodicity of residues of the number of finite labeled topologies. 
%(Russian)
%Modules and linear groups
%Zap. Nauchn. Sem. Leningrad. Otdel. Mat. Inst. Steklov. (LOMI)103(1980), 5--12, 155.
%On the periodicity of residues of the number of finite labeled topologies. 
%Zap. Nauchn. Sem. Leningrad. Otdel. Mat. Inst. Steklov. (LOMI) 103 (1980), 5--12, 155.
%\bibitem{Sharp} H. Sharp, Jr., Quasi-orderings and topologies on finite sets.
%Proc. Amer. Math. Soc.17(1966), 1344--1349.
\bibitem{sloane} N.~Sloane, The On-Line Encyclopedia of Integer Sequences, oeis.org. 



%\emph{On periodicity of residues of the number of finite labeled topologies}, Discrete Math. 277 (2004), no. 1--3, 29--43.
%	\bibitem{BDS Turk} A. Ben Amira,   J. Dammak and H. Si Kaddour, \emph{On a generalization of Kelly's combinatorial lemma}, Turkish J. Math. {\bf 38} (2014), no. 6, 949 -- 964.
	%\bibitem{Sumner} D. Sumner, \emph{Graphs indecomposable with respect to the $X$-join}, Discrete Mathematics, \textbf{6} (1973),  281--298.
	%\bibitem{26} S. M. Ulam, \emph{A collection of Mathematical Problems},  Intersciences Publishers New York, (1960) xiii+150 pp.
%	\bib{Bor80}{article}{
%				author={Z.~I.~Borevich},
%				title={On periodicity of residues of the number of finite labeled topologies},
%				book={
%					title={Modules and linear groups},
%					series={Zap. Nauchn. Sem. LOMI},
%					date={1980},
%					volume={103},
%					pages={5--12},
%					publisher={"Nauka", Leningrad. Otdel.},
%					address={Leningrad},
%					mathnet={http://mi.mathnet.ru/znsl3333},
%					mathscinet={http://mathscinet.ams.org/mathscinet-getitem?mr=618489},
%					zmath={https://zbmath.org/?q=an:0466.05008|0533.05004}
%				},
%				translation={
%					journal={
%						label={jour J. Soviet Math.},
%						date={1984},
%						volume={24},
%						issue={4},
%						pages={391--395}
%					}
%				},
%				webpage={
%					url={https://doi.org/10.1007/BF01094365}
%				}
%			}
\end{thebibliography}
\end{document}



